% Options for packages loaded elsewhere
\PassOptionsToPackage{unicode}{hyperref}
\PassOptionsToPackage{hyphens}{url}
\documentclass[
]{article}
\usepackage{xcolor}
\usepackage{amsmath,amssymb}
\setcounter{secnumdepth}{-\maxdimen} % remove section numbering
\usepackage{iftex}
\ifPDFTeX
  \usepackage[T1]{fontenc}
  \usepackage[utf8]{inputenc}
  \usepackage{textcomp} % provide euro and other symbols
\else % if luatex or xetex
  \usepackage{unicode-math} % this also loads fontspec
  \defaultfontfeatures{Scale=MatchLowercase}
  \defaultfontfeatures[\rmfamily]{Ligatures=TeX,Scale=1}
\fi
\usepackage{lmodern}
\ifPDFTeX\else
  % xetex/luatex font selection
\fi
% Use upquote if available, for straight quotes in verbatim environments
\IfFileExists{upquote.sty}{\usepackage{upquote}}{}
\IfFileExists{microtype.sty}{% use microtype if available
  \usepackage[]{microtype}
  \UseMicrotypeSet[protrusion]{basicmath} % disable protrusion for tt fonts
}{}
\makeatletter
\@ifundefined{KOMAClassName}{% if non-KOMA class
  \IfFileExists{parskip.sty}{%
    \usepackage{parskip}
  }{% else
    \setlength{\parindent}{0pt}
    \setlength{\parskip}{6pt plus 2pt minus 1pt}}
}{% if KOMA class
  \KOMAoptions{parskip=half}}
\makeatother
\usepackage{longtable,booktabs,array}
\usepackage{calc} % for calculating minipage widths
% Correct order of tables after \paragraph or \subparagraph
\usepackage{etoolbox}
\makeatletter
\patchcmd\longtable{\par}{\if@noskipsec\mbox{}\fi\par}{}{}
\makeatother
% Allow footnotes in longtable head/foot
\IfFileExists{footnotehyper.sty}{\usepackage{footnotehyper}}{\usepackage{footnote}}
\makesavenoteenv{longtable}
\usepackage{graphicx}
\makeatletter
\newsavebox\pandoc@box
\newcommand*\pandocbounded[1]{% scales image to fit in text height/width
  \sbox\pandoc@box{#1}%
  \Gscale@div\@tempa{\textheight}{\dimexpr\ht\pandoc@box+\dp\pandoc@box\relax}%
  \Gscale@div\@tempb{\linewidth}{\wd\pandoc@box}%
  \ifdim\@tempb\p@<\@tempa\p@\let\@tempa\@tempb\fi% select the smaller of both
  \ifdim\@tempa\p@<\p@\scalebox{\@tempa}{\usebox\pandoc@box}%
  \else\usebox{\pandoc@box}%
  \fi%
}
% Set default figure placement to htbp
\def\fps@figure{htbp}
\makeatother
\setlength{\emergencystretch}{3em} % prevent overfull lines
\providecommand{\tightlist}{%
  \setlength{\itemsep}{0pt}\setlength{\parskip}{0pt}}
\ifPDFTeX
  \TeXXeTstate=1
  \newcommand{\RL}[1]{\beginR #1\endR}
  \newcommand{\LR}[1]{\beginL #1\endL}
  \newenvironment{RTL}{\beginR}{\endR}
  \newenvironment{LTR}{\beginL}{\endL}
\fi
\usepackage{bookmark}
\IfFileExists{xurl.sty}{\usepackage{xurl}}{} % add URL line breaks if available
\urlstyle{same}
\hypersetup{
  pdftitle={Computer security exercise sessions},
  hidelinks,
  pdfcreator={LaTeX via pandoc}}

\title{Computer security exercise sessions}
\author{}
\date{}

\begin{document}
\maketitle

\section{1 - Buffer Overflow}\label{buffer-overflow}

\subsection{1.1}\label{section}

\pandocbounded{\includegraphics[keepaspectratio,alt={Pasted image 20250608173212.png}]{/home/francesco/Documenti/Documents/Vault/Pasted image 20250608173212.png}}\\
\pandocbounded{\includegraphics[keepaspectratio,alt={Pasted image 20250608173231.png}]{/home/francesco/Documenti/Documents/Vault/Pasted image 20250608173231.png}}

Prolog and Epilogue

BP (base pointer) or FP (frame pointer) -\/-\textgreater{} pointer to a
fixed location on a function\textquotesingle s frame. From which actual
parameters have positive offset, while local variables have negative
offset.\\
SP -\/-\textgreater{} address of the top of the stack\\
IP -\/-\textgreater{} address of the current instruction being executed

\begin{center}\rule{0.5\linewidth}{0.5pt}\end{center}

ESP -\/-\textgreater{} register where the SP is stored (either last cell
or first free cell\textquotesingle s address)\\
EBP -\/-\textgreater{} register where the BP is stored\\
EIP -\/-\textgreater{} register where IP is stored

We racall that as soon as a function call is executed:\\
1. Parameters are pushed on the stack, \texttt{call} assembly
instruction is executed\\
2. Prelude is executed

\textbf{Call}\\
(pushes address of the next instruction to be executed after the called
function returns. This will be the \emph{saved EIP} or \emph{RET})\\
(moves address of the first instruction of the called function into EIP
-\/-\textgreater{} call function)

\textbf{Prolog}:\\
\texttt{pushl\ \%ebp} -\/-\/-\textgreater{} (push BP of caller into
stack. This will be the \emph{saved EBP})\\
\texttt{mov\ \%esp,\ \%esp} -\/-\/-\textgreater{} (save current SP into
EBP register)\\
\texttt{subl\ \$SizeOfLocalVar,\ \%esp} -\/-\textgreater{} (move SP down
in order to make space for local variables)

\textbf{Epilogue}:\\
\texttt{mov\ \%ebp,\ \%esp} (current base is the new top of the stack)\\
\texttt{pop\ \%ebp} (restore the saved EBP to the registry )\\
\texttt{ret} (pop saved EIP and jump there)

Memory

Remember that memory can only be addressed in multiples of a word, which
in our case is \textbf{32 bits}=\textbf{4 bytes}

\begin{longtable}[]{@{}lll@{}}
\toprule\noalign{}
High addresses & & \\
\midrule\noalign{}
\endhead
\bottomrule\noalign{}
\endlastfoot
& Environment variables & \\
& ... & \\
& ... & \\
& ... & -\/- Main Frame \\
& saved EIP & \\
& saved EBP (main) & -\/- \\
& buf{[}28-31{]} & \\
& buf{[}24-27{]} & \\
& buf{[}20-23{]} & \\
& ... & \\
& buf{[}0-3{]} & -\/- vuln frame \\
\textbf{Low addresses} & & \\
\end{longtable}

\pandocbounded{\includegraphics[keepaspectratio,alt={Pasted image 20250608181014.png}]{/home/francesco/Documenti/Documents/Vault/Pasted image 20250608181014.png}}

In order to execute the shellcode, we need to store it into the stack
and then overwrite the saved EIP in order to jump to it. However, we
need to be sure that vuln() will return.\\
What we do:

\begin{itemize}
\tightlist
\item
  Fill the buffer with a string that will make us pass the strcpy check
\item
  Fill the middle part of thebuffer with NOP instructions, since we
  cannot be sure about the address our shellcode will end up
\item
  Fill the last part of the buffer with our shellcode
\end{itemize}

\begin{longtable}[]{@{}lll@{}}
\toprule\noalign{}
High addresses & & \\
\midrule\noalign{}
\endhead
\bottomrule\noalign{}
\endlastfoot
& ... & \\
& ... & \\
& ... & \\
& ... & -\/- Main Frame \\
& \emph{address of A} & \\
& saved EBP (main) & \\
& 0x12 0x34 0x56 0x78 & \\
& 0x90 0x90 0x90 0x90 & \\
A & 0x90 0x90 0x90 0x90 & \\
& "ing!" & \\
& "ht\_K" & \\
& "Knig" & -\/- vuln frame \\
\textbf{Low addresses} & & \\
\end{longtable}

\subsection{1.2}\label{section-1}

Buffer overflow: \texttt{scanf} at line 08 will allow us to insert an
exploit that will overwrite the EIP, since the buffer length is 8 but
the scanf reads 25 characters\textquotesingle{} string.

For our exploit to work, we need to be sure vuln() correctly returns and
it is not aborted, thus:\\
we need to be sure \texttt{str\ =\ DAVINCI}. However, we notice that the
buffer its reversed before the check.

\begin{longtable}[]{@{}lll@{}}
\toprule\noalign{}
High addresses & & \\
\midrule\noalign{}
\endhead
\bottomrule\noalign{}
\endlastfoot
& ... & \\
& ... & \\
& ... & -\/- main frame ends here \\
& saved EIP & \\
& saved EBP & \\
& str{[}4-7{]} & \\
& str{[}0-3{]} & -\/-\/- vuln frame ends here \\
\textbf{Low addresses} & & \\
\end{longtable}

Situation in memory we desire to have right before the check:

\begin{longtable}[]{@{}lll@{}}
\toprule\noalign{}
High addresses & & \\
\midrule\noalign{}
\endhead
\bottomrule\noalign{}
\endlastfoot
{[}24-27{]} & 0x38 0x37 0x36 0x35 & \\
\textbf{A} & 0x34 0x33 0x32 0x31 & \\
& ... & -\/- main frame ends here \\
& \emph{address of A} & \\
{[}8-11{]} & AAAA & \\
{[}4-7{]} & "NCI:" & \\
{[}0-3{]} & "DAVI" & -\/-\/- vuln frame ends here \\
\textbf{Low addresses} & & \\
\end{longtable}

That is in a normal setting the desired exploit is:
\texttt{DAVINCI:AAAA\textless{}ptr\_to\_shellcode\textgreater{}\textbackslash{}x31\textbackslash{}x32\textbackslash{}x33\textbackslash{}x34\textbackslash{}x35\textbackslash{}x36\textbackslash{}x37\textbackslash{}x38}

Code direction

The code is inserted in the usual way we insert stuff in the stack:
starting from the lowest address and going up. So, the address of the
shellcode with which we overwrite the saved EIP is actually the lowest
one (A)

\texttt{strrev} working

From what we can infer, strevv reverses the string passed as an
argument: it\textquotesingle s like reading the string from the end and
writing it on the buffer as it was an input

Little endian?

Why we are now using little endian (stuff is mirrored inside the cell)
and in the first exercise we didn\textquotesingle t?

Little endian and strings

Notice that even when using little endian, strings are represented
normally

Missing terminator

Why don\textquotesingle t we see a terminator in the stack? In principle
the \texttt{scanf} should automatically add the terminator after the
input (in this case it actually will take 24 characters as input and
then add the terminator, in order to reach the required 25 characters)

However, in our case the buffer is reversed. Thus, we need to reverse
our string. The actual exploit will be:\\
\texttt{\textbackslash{}x38\textbackslash{}x37\textbackslash{}x36\textbackslash{}x35\textbackslash{}x34\textbackslash{}x33\textbackslash{}x32\textbackslash{}x31\textless{}ptr\_to\_shellcode\_inv\textgreater{}AAAA:ICNIVAD}

The situation in memory we need to have before the \texttt{strrev}:

\begin{longtable}[]{@{}lll@{}}
\toprule\noalign{}
High addresses & & \\
\midrule\noalign{}
\endhead
\bottomrule\noalign{}
\endlastfoot
{[}24-27{]} & "IVAD" & \\
\textbf{A} & ":ICN" & \\
& ... & -\/- main frame ends here \\
& AAAA & \\
{[}8-11{]} & address of A & \\
{[}4-7{]} & 0x31 0x32 0x33 0x34 & \\
{[}0-3{]} & 0x35 0x36 0x37 0x38 & -\/-\/- vuln frame ends here \\
\textbf{Low addresses} & & \\
\end{longtable}

\subsection{1.3}\label{section-2}

By analysing the code we see:

\begin{itemize}
\tightlist
\item
  \textbf{vulnerability}: \texttt{scanf} at line 17. Since there is no
  check on maximum string length, we can perform a stack smashing by
  overwriting the saved EIP
\item
  there is a parameter to the \texttt{vuln()} function
\item
  the parameter, \texttt{n}, is then used to check if we should enter
  the if. So it must be overwritten accordingly to the check, otherwise
  we won\textquotesingle t reach the vulnerability. -\/-\textgreater{}
  \textbf{Condition 1:} 0\textless parameter\textless16

  \begin{itemize}
  \tightlist
  \item
    we need to avoid entering the \texttt{if} at line 14, which will
    lead to \texttt{abort()} -\/-\textgreater{} \textbf{Condition 2}: we
    need to put "HACK" in \texttt{s.buf}and we put something different
    from "X" in \texttt{s.buf{[}14{]}}. (notice that in the original
    text there is an error, the operator should be
    \texttt{\textbar{}\textbar{}} not \texttt{\&\&})
  \end{itemize}
\item
  avoid entering if at line 18, which will cause abort.
  -\/-\textgreater{} \textbf{Condition 3}: don\textquotesingle t change
  the content of \texttt{s.sparrow}
\end{itemize}

Stack at line 12:

struct is not dynamically allocated but statically, so it is allocated
in the stack.

\begin{longtable}[]{@{}llll@{}}
\toprule\noalign{}
\textbf{High addresses} & \textbf{What} & & \\
\midrule\noalign{}
\endhead
\bottomrule\noalign{}
\endlastfoot
& & & \\
& & argc & -\/- main\textquotesingle s frame end \\
& & saved EIP & \\
& & saved EBP & \\
& sparrow & 0xAD 0xAA 0xAA 0xBA & \\
& & tmp{[}12-15{]} & \\
& & ... & \\
& & tmp{[}0-3{]} & \\
& & buf{[}12-15{]} & \\
& & ... & \\
& & buf{[}0-3{]} & -\/- vuln\textquotesingle s frame end \\
& & & \\
& & & \\
\end{longtable}

Struct static allocation

If a function has a struct as local variable and it is statically
allocated, it is positioned in the stack as the sum of its parts. As
always, it is allocated in "reverse order", so we position its fields on
the stack by starting from the last field and go back towards the first

The shellcode is: \texttt{0x51\ 0x52\ 0x5a\ 0x5b}. Only 4 bytes, we can
easily insert it in the stack.

\begin{longtable}[]{@{}llll@{}}
\toprule\noalign{}
\textbf{High addresses} & \textbf{What} & & \\
\midrule\noalign{}
\endhead
\bottomrule\noalign{}
\endlastfoot
& & & \\
& argc & \texttt{0x00\ 0x00\ 0x00\ 0x02} & -\/- main\textquotesingle s
frame end \\
& & \emph{address A} & \\
& & saved EBP & \\
& sparrow & \texttt{0xAD\ 0xAA\ 0xAA\ 0xBA} & \\
& tmp{[}12-15{]} & \texttt{0x5b\ 0x5a\ 0x52\ 0x51} & \\
& & \texttt{0x90\ 0x90\ 0x90\ 0x90} & \\
& & \texttt{0x90\ 0x90\ 0x90\ 0x90} & \\
Address A & tmp{[}0-3{]} & \texttt{0x90\ 0x90\ 0x90\ 0x90} & \\
& buf{[}12-15{]} & "X" "X" "X" not rel & \\
& & not relevant & \\
& & not relevant & \\
& buf{[}0-3{]} & "HACK" & -\/- vuln\textquotesingle s frame end \\
\textbf{Low addresses} & & & \\
\end{longtable}

\pandocbounded{\includegraphics[keepaspectratio,alt={Pasted image 20250612234139.png}]{/home/francesco/Documenti/Documents/Vault/Pasted image 20250612234139.png}}

\texttt{System()}function

\texttt{\textgreater{}\ int\ system(const\ char\ *\_\_command\_\_);\ }\strut \\
command: A pointer to a null-terminated string that contains the command
we want to execute.

In our case, the command we want to execute is \texttt{/bin/secret}. For
this reason, we need to have this null-terminated string somewhere in
memory, and make system use a pointer to it as parameter.

\begin{longtable}[]{@{}llll@{}}
\toprule\noalign{}
\textbf{High addresses} & \textbf{What} & & \\
\midrule\noalign{}
\endhead
\bottomrule\noalign{}
\endlastfoot
& & \textbf{address A} & The address to the command \\
add1 & & \textbf{exit} & -\/- main\textquotesingle s frame end \\
add2 & former saved EIP & \textbf{\texttt{0xf7e38da0}} & system addr \\
& & saved EBP (not relevant) & \\
& sparrow & \texttt{0xAD\ 0xAA\ 0xAA\ 0xBA} & \\
& tmp{[}12-15{]} & \texttt{0x5b\ 0x5a\ 0x52\ 0x51} & \\
& & \textbf{"ret\textbackslash0"} & \\
& & \textbf{"/sec"} & \\
Address A & tmp{[}0-3{]} & \textbf{"/bin"} & \\
& buf{[}12-15{]} & "X" "X" "X" not rel & \\
& & not relevant & \\
& & not relevant & \\
& buf{[}0-3{]} & "HACK" & -\/- vuln\textquotesingle s frame end \\
\textbf{Low addresses} & & & \\
\end{longtable}

What happens as soon as the function returns?

\begin{itemize}
\tightlist
\item
  system\textquotesingle s address is the saved EIP, so the \texttt{ret}
  instruction will pop its address and load it into EIP. Since the
  loaded address is the address of a function (\texttt{system()} in this
  case), the prelude is done: the old EBP is pushed on the stack, thus
  at this point the situation is the following.
\end{itemize}

\begin{longtable}[]{@{}llll@{}}
\toprule\noalign{}
\textbf{High addresses} & \textbf{What} & & \\
\midrule\noalign{}
\endhead
\bottomrule\noalign{}
\endlastfoot
& parameter & \textbf{address A} & The address to the command \\
& EIP (corresponds to add1) & \textbf{exit} & -\/-
main\textquotesingle s frame end \\
& EBP (corresponds to add2) & new EBP & system addr \\
& & & \\
\end{longtable}

As we see, during the system() execution, we have the frame we needed
like to have: exit as return address and address A as function
parameter. This takes as to the following important point:

Where to put parameters and return address of called function after
overwriting saved EIP

When we overwrite the EIP with the address of a function, we need to
rember that the \texttt{ret} will pop that address and then the prelude
of the function will be executed, pushing the EBP on the stack: that is,
the cell of the overwritten EIP is where the EBP of the called function
will be placed. So the simply: \textbf{when we overwrite a saved EIP,
that cell will contain the saved EBP of the called function, the cell
above (toward higher addresses) will contain the saved EIP of the called
function and the cell above again will contain the parameter to that
function}\\
(see example above)

\section{2 - format strings
vulnerability}\label{format-strings-vulnerability}

Format strings vulnerability basics

In format string vulnerability we exploit \texttt{printf(x)} where we
are able to define \texttt{x}.\\
We can start from how \texttt{printf} works:

\begin{itemize}
\tightlist
\item
  According to the calling convention, the parameters for a
  \texttt{printf} call are pushed on the stack by the caller. The other
  arguments, such a pointer the \emph{format string} itself, are pushed
  on the stack. Usually the format string itself is in some position in
  the stack, and what\textquotesingle s pushed as argument a pointer to
  it.\\
  For example, if we have something like
  \texttt{printf("AAAA\ \%x\ \%x",\ value1,\ value2)}. The stack will be
\end{itemize}

\begin{longtable}[]{@{}ll@{}}
\toprule\noalign{}
High addresses & \\
\midrule\noalign{}
\endhead
\bottomrule\noalign{}
\endlastfoot
& "AAAA" \\
& "\%x\%x" \\
& ... \\
& value2 \\
& value1 \\
& pointer to format string \\
& EIP \\
& EBP \\
\end{longtable}

Thus, inserting \texttt{\%x} in the \texttt{printf} without any
corresponding parameter will result to printing whatever is on the
stack, where it would expect its arguments to be.

Also, we can use special \texttt{printf} operator to print an arbitrary
number of characters and to write something in the stack:

\begin{itemize}
\tightlist
\item
  \%n = write, in the address pointed to by the argument, the
  \emph{number of chars (bytes) printed so far}. That is, it takes
  whatever is on the stack corresponding to its position (it works as
  \%x, but the latter simply prints its argument out, \%n does something
  else with it), interprets it as an address, goes to that location and
  write in that location the number of char printed so far. We need:
  \textbf{address} on the stack, corresponding to a \textbf{target} cell
  where to write into.
\item
  \texttt{\%{[}k{]}c}: prints out a number k of characters.\\
  We can use this operator in order to easily decide the \emph{number of
  chars (bytes) printed so far}, which is useful for the \texttt{\%n}
  operator. In practice, the \emph{number of chars (bytes) printed so
  far} is a very big number which corresponds to the \textbf{shellcode}
  we want to write on the \textbf{target}.
\item
  \texttt{\%{[}k{]}\$x}: allows to print the k-th parameter. For example
  \texttt{\%2\$x} write the same as the second \texttt{\%x} in
  \texttt{\%x\ \%x}.\\
  We can use this operator to understand the \textbf{pos} between where
  the format string is located on the stack and the top of the stack
  (where we are.
\item
  Combining the first and the third, we get \texttt{\%{[}k{]}\$n}, which
  writes in the memory location whose address it finds in position k on
  the stack, the \emph{number of chars (bytes) printed so far}
\end{itemize}

The attack: we want to write in the \textbf{target} memory cell, whose
\textbf{address} is at distance \textbf{pos} from the top of the stack,
a \textbf{shellcode}.\\
The exploit is a format string like:\\
\texttt{address\ \%{[}K{]}c\ \%{[}pos{]}n}

What happens if we pass it to a \texttt{printf}.

\begin{itemize}
\tightlist
\item
  The format string itself will end up in some position in the stack.
\item
  the \texttt{\%{[}pos{]}n} operator will at the \textbf{address}
  located at \textbf{pos} position in the stack (which we want to be our
  \textbf{target}), the number of characters \texttt{length(address)+K},
  which we need to be our \textbf{shellcode}.
\end{itemize}

By carefully choosing the parameters of the exploit, we can write what
we want in the target. Usually we can chose the saved EIP to be our
target. So our \textbf{address}\\
Since the \textbf{address} could be very big, and we need to write a
number of characters similar to \textbf{address} (using the \%\$c
operator), we perform two distinct writings, by splitting the
\textbf{target} in two parts. After this consideration and some maths
about what parameters to use, here is how the final exploit should be
built:

\pandocbounded{\includegraphics[keepaspectratio,alt={Pasted image 20250617191032.png}]{/home/francesco/Documenti/Documents/Vault/Pasted image 20250617191032.png}}\\
\pandocbounded{\includegraphics[keepaspectratio,alt={Pasted image 20250617191041.png}]{/home/francesco/Documenti/Documents/Vault/Pasted image 20250617191041.png}}

\end{document}
